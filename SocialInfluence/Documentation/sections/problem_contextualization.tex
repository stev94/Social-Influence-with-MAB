\newpage
\section{Problem contextualization}

The problem we are trying to tackle is how to maximize social influence in a given network given some constraints.
These constraints are mainly regarding a budget, that is, the maximum value of nodes that can be bought inside the network.
The objective is to reach the highest number of users with the given budget.
This is due the fact that each node that is acquired, and is put in the Seed Set, is capable of influencing with a probability its neighbouring nodes, so all the nodes reached by an outgoing edge.
This is done in a cascade fashion, so if node 2 is bought, and node 2 influences node 3 at time t=1, then node 3 may as well do the same to the nodes it can reach at time t=2.
This process is over when every active node, which are the nodes influenced, attempted the influence process and terminated.
The influence spread, which is usually what we want to maximize, is the number of nodes infleunced by the this process starting from a selected Seed set.
Inside the graph, each node has features, and all of these features are used to build the final probability that represents an edge between two nodes.
In order to fully estimate the effect that buying a node has, we have to simulate the activation of the edges in an iterative way.
To do so, Monte Carlo simulations is used since is a good heuristic for the estimation of the influence spread.

